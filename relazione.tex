\documentclass[a4paper, openany, 10pt]{book}


\usepackage[utf8]{inputenc}
\usepackage[T1]{fontenc}
\usepackage[italian]{babel}
\usepackage{concrete}
\usepackage{beton}
\usepackage{concmath}
\usepackage{ccfonts}
\renewcommand{\bfdefault}{sbc} 

%\usepackage{uarial}
%\renewcommand{\familydefault}{\sfdefault}
%\usepackage{blindtext}
%\usepackage{amsmath}
%\usepackage{amssymb}
%\usepackage{amsfonts}
%\usepackage{euler}
%\usepackage{amsthm}
%\usepackage{mathtools}

\pagestyle{headings}


\begin{document}

\title{Relazione basi di dati}
\author{Cristian Maschio\\Andrea Favero}
\date{1 dicembre 2016}
\maketitle

\tableofcontents

\chapter{Abstract}
Il Dipartimento di Matematica Tullio Levi Cività è una struttura dell'Università degli Studi di Padova
che promuove e coordina le attività di ricerca nella Matematica e nell'Informatica. Esso coordina i
corsi di studio di Matematica ed Informatica (sia a livello di Laurea Triennale che Magistrale) e,
%si occupa del loro insegnamento anche nei corsi di studio di altri settori scientifici disciplinari (SSD) che, sono gestiti da altri dipartimenti dell'ateneo 
si occupa anche dell'insegnamento di tali scienze nei corsi di studio di diversi settori scientici, che
sono gestiti da altri dipartimenti (per esempio si occupa dell'insegnamento 
Analisi Matematica per il corso di studi in Ingegneria Aerospaziale che è gestito dal Dipartimento di Ingegneria Industriale).
Il dipartimento gestisce la Scuola di Dottorato in Scienze Matematiche e partecipa al dottorato in
Informatica. 

Al dipartimento afferisce un vasto numero di personale che si ripartisce in docenti, docenti esterni,
assegnisti, borsisti, collaboratori, dottorandi, personale tecnico amministrativo e ospiti (per esempio
professori in visita da altre università).

La sede del dipartimento è la Torre Archimede che si trova in via Trieste 63. Il dipartimento possiede
anche delle aule e due laboratori informatici presso il Plesso Paolotti in via Belzoni 7 e via Luzzati
11.



\end{document}
